
\chapter*{前書き}

この本は、単純型付きラムダ計算からパラメトリック多相入りの型システム程度の範囲での型推論について、
体系的かつ理論と実装に一対一の対応が現れる形でまとめた本です。最後まで読めばHindley-Milner
の型推論アルゴリズム$\mathcal W$の理論と実装を理解できるということを目標としています。

型推論とは、静的型付きのプログラミング言語において、
その型情報の欠けたプログラムから変数や式の型を推論する仕組みのことです。
Hindley-Milnerの型推論アルゴリズムは、
多くのプログラミング言語処理系における型推論の基礎となっているアルゴリズムです。

そもそも、Hindley-Milner型推論アルゴリズムを学ぶことにはどのような意味があるのでしょうか。
著者は、この問いに対する答えは二つあると考えています。

まず一つは、型推論を持つ静的型付けのプログラミング言語の処理系を作る上で、
避けては通れない道であるということです。
しかし、それだけの理由で型推論を学ぶ必要がある人はごく僅かしかいません。

もう一つは、型推論の仕組みを知ることで、
型推論の機能を持つ言語処理系の型に関するメッセージが圧倒的に理解しやすくなるということです。

OCamlやHaskell、その他類似の型推論機能を持つ言語を使っていて、
型エラーの意味する所が良く分からないということがあると思います。
そのエラーは、型推論の過程で整合性が取れないと分かった部分を示しているのです。
よって、型推論アルゴリズムを学ぶことで、型エラーに対する理解がより一層深くなると考えられます。
また、Haskellではトップレベルの束縛に型を書くのが正しい習慣とされていますが、
それも型推論の仕組みを逆手に取り、分かりづらい型エラーを抑制する意味があります。

このような理由から、型推論を持つプログラミング言語を使っているプログラマにとって、
型推論を学ぶことはとても大きな意味があると考えられます。

\section*{対象読者}

読者は関数型プログラミングやOCamlのエキスパートである必要はありません。
しかし、ある程度HaskellやOCamlやその他類似の言語での関数型言語らしいプログラムの書き方を
理解していることを前提としています。

\section*{型推論の実装について}

本書で用いる型推論の実装は関数型プログラミング言語OCaml 3.12.1、
OCamlの標準ライブラリ、Jane Street's coreライブラリによって記述されています。
この型推論の実装は全て著者の坂口によるものです。

本書の内容には、全てのソースコードは含めないようにしています。
全てのソースコードは、以下のURIからダウンロードできます。

\texttt{http://github.com/pi8027/typeinfer}

\section*{謝辞}



