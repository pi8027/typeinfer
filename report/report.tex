\documentclass[a4paper]{jsarticle}

\usepackage{amsmath}
\usepackage{amssymb}
\usepackage{cite}
\usepackage{proof}
\usepackage[dvipdfmx]{graphicx}

\bibliographystyle{plain}

\title{情報特別演習Iレポート}
\author{情報学群情報科学類 201111365 坂口和彦}

\begin{document}

\maketitle

\section{概要}

私の演習テーマは、プログラミング言語OCamlによるHindley-Milnerの型推論アルゴリズムの実装です。
また、型推論の実装だけではなく、
その背景にある理論までを含めた型推論の解説本「Algorithm $\mathcal W$入門」を執筆しました。

本のタイトルの「Algorithm $\mathcal W$」とは、Hindley-Milnerの型推論アルゴリズムのことです。
Robin Milnerの論文\cite{milner1978}で記号$\mathcal W$がこの意味で使われて以来、今でもこのように呼ばれています。

最終的な成果物としてできた本「Algorithm $\mathcal W$入門」は、B5サイズ48ページの本です。
印刷業者を利用して250部印刷しました。

\section{経緯}

演習の経緯を以下に簡単にまとめます。

まず演習テーマの決定ですが、私は前から型推論アルゴリズムに興味があり、
学習したことなどを本としてまとめるのも習慣化していたため、
演習テーマのほとんどは元々自分で決めていたものです。
私は2010年にも一度型推論に関する勉強をし、共著で本を執筆していました。
しかし、その自分が書いた部分にまだ不満があったため、
それらの不満を解決することが一つの大きな目標となっていました。

私はこの演習の最初の段階として、本の中での解説の組み立て方と実装方法について考えました。
Hindley-Milnerの型推論アルゴリズムは非常に複雑ですが、そこに辿りつくまでの道筋を適切に作り、
適切な段階に分けてやることでその理解はとても容易なものとなります。
よって、解説の組み立て方も実質的にはこの段階の切り分けをする作業となりました。
実装も、その段階に合わせてそれぞれ作っていくことにしました。

8月頃までは、全ての型推論の実装と本の最初の部分の執筆に取り組みました。

9月頃からの作業のほとんどは、本の執筆でした。
新しい内容を書きつつ、レビュアーの方々の意見などを元に修正を行っていきました。
また、必要であれば実装の修正なども行いました。

11月下旬頃からは、本としての見た目の調整、表紙の作成、校正、発表資料の作成などをしました。
12月中旬には本が完成し、印刷業者にPDFを送ることができました。

コミックマーケット81で、Algorithm $\mathcal W$入門を頒布しました。

最終的には、実装や本の完成度も自分自身の演習に対する取り組みも、十分なものであったと考えています。

\section{環境の整備}

本の執筆作業において、その環境、即ち使うソフトウェアや設定がとても重要になります。

私はこの演習を進める上で、環境を適切に作ることで快適に執筆作業を進めることに成功しました。
そのことについて少し書きます。

まず、組版には\LaTeX{}を用いました。エディタはEmacs(YaTeX)と時々Vimを使いました。
\LaTeX{}を利用していると編集をしながら出力を見たいということがあると思いますが、
継続ビルドのツールとしてOMakeを使い、
ファイルが変更されたときに読み直すようなPDFビューア(evince, okularなど)を使うことで実現できました。

執筆作業ではPDFビューアとエディタとOMakeの画面を同時に表示させておく必要があり、
普段使っているノートPCの狭いディスプレイでは工夫をしないと作業が困難でした。
しかし、私が普段使っているウィンドウマネージャxmonadを少し改造することでこの問題を解消できました。

原稿や実装のソースコードの管理のため、バージョン管理システムGitを用いました。
Gitサーバとしては、私が借りているさくらのVPSとGitHubを用いました。

Gitには、コミットやプッシュのタイミングで自動で指定した処理を走らせるフックの機能があります。
このフック機能を用いることで、変更をプッシュした時にサーバ側で本のPDFを自動生成し、
HTTPサーバを介して参照できるようにしました。このようにすることで、
Algorithm $\mathcal W$入門のレビュアー向けに最新版のPDFを少ない手間で提供できるようになりました。
また、フックにはTwitterを用いたレビュアーへの通知も含まれています。

課題管理にはGitHubのissue trackerを用いました。レビュアーからの意見なども、これで管理しました。

\section{実装}

型推論の実装には、プログラミング言語OCamlと代替標準ライブラリJane Street's Coreを使いました。

上でも複数の実装が存在することを伸べましたが、最終的には以下の4種類の実装に落ち着きました。

\begin{description}
  \item[単純型付き$\lambda$計算の型推論]
    とても簡単な型付き言語の体系である単純型付き$\lambda$計算の型推論です。
  \item[多相性を含む体系の型推論]
    MLのlet項を含む体系の型推論は、letを全て簡約した上でそれを除いた体系への
    型推論アルゴリズムを適用することで実現できると知られています。

    多相性の導入部分として、これを採用しました。
  \item[パラメトリック多相型システムの型推論]
    多相性を置き換えの意味ではなく、パラメトリック多相型の意味で扱うようにし、
    その型システムに対する型推論アルゴリズムを記述しました。
  \item[Algorithm $\mathcal W$]
    パラメトリック多相型システムの型推論を変形することにより、
    本の上で目標としている$\mathcal W$が得られます。これを記述しました。
\end{description}

それぞれの実装にはocamllexとocamlyaccによるパーサなども含まれていて、150行から200行程度で記述されています。

\section{本の構成}

完成した時点でのAlgorithm $\mathcal W$入門の構成は、図\ref{fig:book_graph}のようになりました。

\begin{figure}[htbp]
\begin{center}
\includegraphics[width=160mm]{graph.ps}
\end{center}
\label{fig:book_graph}
\caption{Algorithm $\mathcal W$入門の構成}
\end{figure}

図\ref{fig:book_graph}の有向グラフの頂点はそれぞれが本の中での話題のまとまりに対応し、
辺はそのまとまりの間にある依存関係を表しています。
(矢印の先の頂点が元の頂点に依存しているという風に読みます。)

赤い頂点は背景となる理論の話であることを表し、
青い頂点は型推論アルゴリズムの話であることを表しています。

Algorithm $\mathcal W$入門は、3つの章から成ります。

1章では、型推論の対象となる言語や型システムの基礎を解説しています。

2章では、1章の最後に出てくる単純型付き$\lambda$計算という型システムに対する型推論アルゴリズムを、
連立方程式の生成とそのソルバに分けて解説しています。

3章では、多相性の導入とパラメトリック多相、多相型システムに対する型推論アルゴリズム、
とくに$\mathcal W$について解説しています。

% 参考文献

\nocite{*}
\bibliography{report}

\end{document}
