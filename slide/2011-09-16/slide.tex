\documentclass[cjk, 14pt]{beamer}

\usepackage{amsmath}
\usepackage{amssymb}
\usepackage{listings}
\usepackage{ascmac}
\usepackage{framed}
\usepackage{tikz}

\usetheme{Copenhagen}
\usecolortheme{seahorse}
\useinnertheme{rounded}
\useoutertheme{shadow}

\setbeamercovered{transparent}

\lstset{
  basicstyle=\ttfamily,
  basicstyle=\color{white}\ttfamily\scriptsize,
  backgroundcolor=\color{black},
  columns=[l]{fullflexible}
}

\title{Algolithm $\mathcal W$ 入門}
\subtitle{分かりやすい型推論実装入門本}
\author{坂口和彦}
\institute{筑波大学 情報学群 情報科学類 B1}
\date{2011/09/16}


\begin{document}

\begin{frame}{テーマ : Algorithm $\mathcal W$ 入門}

 \begin{itemize}
  \item Hindley-Milner の型推論アルゴリズム(Algorithm $\mathcal W$)の実装
  \item 型推論の解説本 (60ページ程度の予定)
 \end{itemize}
 を作る

  \pause

 \begin{itemize}

  \item レビュアー募集中

  \pause

  \item \texttt{https://github.com/pi8027/typeinfer}
 \end{itemize}

\end{frame}

\end{document}
